\documentclass[fleqn]{llncs}
\usepackage[utf8]{inputenc}
\usepackage[T1]{fontenc}

% nice code font
\usepackage[scaled=0.8]{beramono}

% math stuff
\usepackage{amsmath}
\usepackage{amssymb}
\usepackage{mathtools}

\usepackage{xspace}

% nice citations
\usepackage[numbers]{natbib}

% clickable links and cross-references
\usepackage{hyperref}
\hypersetup{hidelinks,
    colorlinks=true,
    allcolors=blue,
    pdfstartview=Fit,
    breaklinks=true}

% easy cross-references with \cref
\usepackage[capitalise,nameinlink]{cleveref}

\pagestyle{plain}

\newcommand{\Cuvee}{\textsc{Cuvée}\xspace}

\title{\Cuvee: Blending SMT-LIB with \\ Programs and Weakest Preconditions}
\author{Gidon Ernst}
\institute{LMU Munich, Germany, \email{gidon.ernst@lmu.de}}


\begin{document}
\maketitle

\begin{abstract}
\Cuvee is a program verification tool that reads SMT-LIB-like input files where expressions may additionally contain weakest precondition operators over abstract programs.
\Cuvee translates such inputs into first-order SMT-LIB by symbolically executing these programs.
The input format used by \Cuvee is intended to bring different verification communities closer together and achieve a similar unification of tools for that for example synthesize loop summaries.
A notable technical aspect of \Cuvee itself is the consequent use of loop pre-/postconditions instead of invariants, and we demonstrate how this lowers the annotation burden on some simple while programs.
\end{abstract}

\begin{keywords}
Program Verification, SMT-LIB, Weakest Precondition
\end{keywords}

\section{Introduction}

Intermediate verification languages and tools such as Boogie, Why3, and Viper
have had a significant impact on the state-of-the-art of (deductive) program verification.
At the annual competition on interactive program verification VerifyThis~\cite{},
tools like these are put to practice on small but intricate verification problems.

SMT-LIB~\cite{} is a standardized interchange format for verification tasks in first-order logic
that is widely used in many different application domains such as constraint-solving, program verification, and model-checking.
Many mature tools are available \cite{}, and the annual SMT-COMP evaluates and compares their performance on benchmark problems.
Part of the success of the SMT-LIB format is its regular syntax and its precise standardization.

SV-COMP~\cite{} is 


In comparison to existing intermediate verification languages such as Boogie, Why3, and Viper,
it is more light-weight and aims to be easier to implement and process automatically.
The input format used by \Cuvee is intended to bridge the gap between automated theorem proving,
and between the goals of SV-COMP (fully automatic verification of C programs)
and the VerifyThis competition (

\end{document}
